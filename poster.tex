% Gemini theme
% https://github.com/anishathalye/gemini

\documentclass[final]{beamer}

% ====================
% Packages
% ====================

\usepackage[T1]{fontenc}
\usepackage{lmodern}
\usepackage[size=a0,orientation=portrait,scale=1.45]{beamerposter}
\usetheme{gemini}
\usecolortheme{unamur}
\usepackage{graphicx}
\usepackage{svg}
\usepackage{booktabs}
\usepackage{tikz}
\usepackage{pgfplots}
\pgfplotsset{compat=1.14}
\usepackage{anyfontsize}

% ====================
% Lengths
% ====================

% If you have N columns, choose \sepwidth and \colwidth such that
% (N+1)*\sepwidth + N*\colwidth = \paperwidth
\newlength{\sepwidth}
\newlength{\colwidth}
\setlength{\sepwidth}{0.025\paperwidth}
\setlength{\colwidth}{0.3\paperwidth}

\newcommand{\separatorcolumn}{\begin{column}{\sepwidth}\end{column}}

% ====================
% Title
% ====================

\title{Some fancy title: followed by some more text}

\author{Hosam Elkoulak}

\institute[shortinst]{\inst{1} University of Namur}

% ====================
% Footer (optional)
% ====================

\footercontent{
  \href{https://www.example.com}{https://www.example.com} \hfill
  ABC Conference 2025, New York --- XYZ-1234 \hfill
  \href{mailto:alyssa.p.hacker@example.com}{alyssa.p.hacker@example.com}}
% (can be left out to remove footer)

% ====================
% Logo (optional)
% ====================

% use this to include logos on the left and/or right side of the header:
% \logoright{\includegraphics[height=7cm]{logo1.pdf}}
% \logoleft{\includegraphics[height=7cm]{logo2.pdf}}

% ====================
% Body
% ====================

\begin{document}

  \begin{frame}[t]
    \begin{columns}[t]
      \separatorcolumn

      \begin{column}{\colwidth}

        \begin{block}{TCPLS Handshake}

          A typical TCPLS handshake starts with a TCP handshake followed by a TLS 1.3 handshake.

          \begin{figure}
            \centering
            \includesvg[width=\textwidth]{tcpls_hs}  % Without the .svg extension
            \caption{TCPLS handshake}
            \label{fig:tcpls-hs}
          \end{figure}

          TCPLS supports joining additional TCP connections to a TCPLS session.
          To join a new TCP connection, the client sends a fake client hello
          containing one of the tokens received during the TCPLS handshake.

        \end{block}

        \begin{block}{TCPLS Record}

          TCPLS leverages the TLS records to transport data encrypted and authenticated.

          \begin{figure}
            \centering
            \includesvg[width=\textwidth]{tcpls_record}
            \caption{TCPLS Record}
            \label{fig:tcpls-record}
          \end{figure}


        \end{block}
        \begin{block}{TCPLS Transport Services}

          By leveraging TCPLS records and extensions, we can design new
          modern transport services atop the combination of TCP and TLS. Those services are built atop the TCP
          byte stream, and as such they avoid interferences of middleboxes that modify the TCP header.

          \begin{itemize}
            \item \textbf{Stream Multiplexing}
            \item \textbf{Multipath}
            \item \textbf{Failover}
          \end{itemize}
        \end{block}
        \begin{figure}
          \centering
          \includesvg[width=\textwidth]{unamur}
          \label{fig:unamur}
        \end{figure}
        \begin{figure}
          \centering
          \includesvg[width=\textwidth]{cyberwal}
          \label{fig:cyberwal}
        \end{figure}

      \end{column}

      \separatorcolumn

      \begin{column}{\colwidth}
        \begin{block}{Multipath}

          By extending the TLS handshake, TCPLS allows joining several TCP connections to a
          single TCPLS session to support \textbf{connection migration},
          \textbf{stream steering} and \textbf{bandwidth aggregation}.
        \begin{figure}
          \centering
          \includesvg[width=\textwidth]{multipath}
          \caption{Multipath}
          \label{fig:multipath}
        \end{figure}


        \end{block}

        \begin{block}{Stream Multiplexing}
          Streams can be multiplexed across the available TCP connections of the TCPLS session.
          \begin{figure}
            \centering
            \includesvg[width=\textwidth]{stream_multiplex}
            \caption{Stream Multiplexing}
            \label{fig:stream-multiplex}
          \end{figure}
        \end{block}



        \begin{block}{Failover}
          When a TCP connection fails, e.g., due to middleboxes
          or network outages, TCPLS leverages its joining mechanism to re-
          cover the session over another TCP connection.
           \begin{figure}
            \centering
            \includesvg[width=\textwidth, inkscapelatex=false]{failover}
            \caption{Failover}
            \label{fig:failover}
          \end{figure}

        \end{block}

      \end{column}

      \separatorcolumn

      \begin{column}{\colwidth}

        \begin{exampleblock}{A highlighted block containing some math}

          A different kind of highlighted block.

          $$
          \int_{-\infty}^{\infty} e^{-x^2}\,dx = \sqrt{\pi}
          $$

          Interdum et malesuada fames $\{1, 4, 9, \ldots\}$ ac ante ipsum primis in
          faucibus. Cras eleifend dolor eu nulla suscipit suscipit. Sed lobortis non
          felis id vulputate.

          \heading{A heading inside a block}

          Praesent consectetur mi $x^2 + y^2$ metus, nec vestibulum justo viverra
          nec. Proin eget nulla pretium, egestas magna aliquam, mollis neque. Vivamus
          dictum $\mathbf{u}^\intercal\mathbf{v}$ sagittis odio, vel porta erat
          congue sed. Maecenas ut dolor quis arcu auctor porttitor.

          \heading{Another heading inside a block}

          Sed augue erat, scelerisque a purus ultricies, placerat porttitor neque.
          Donec $P(y \mid x)$ fermentum consectetur $\nabla_x P(y \mid x)$ sapien
          sagittis egestas. Duis eget leo euismod nunc viverra imperdiet nec id
          justo.

        \end{exampleblock}

        \begin{block}{Nullam vel erat at velit convallis laoreet}

          Class aptent taciti sociosqu ad litora torquent per conubia nostra, per
          inceptos himenaeos. Phasellus libero enim, gravida sed erat sit amet,
          scelerisque congue diam. Fusce dapibus dui ut augue pulvinar iaculis.

          \begin{table}
            \centering
            \begin{tabular}{l r r c}
              \toprule
              \textbf{First column} & \textbf{Second column} & \textbf{Third column} & \textbf{Fourth} \\
              \midrule
              Foo                   & 13.37                  & 384,394               & $\alpha$        \\
              Bar                   & 2.17                   & 1,392                 & $\beta$         \\
              Baz                   & 3.14                   & 83,742                & $\delta$        \\
              Qux                   & 7.59                   & 974                   & $\gamma$        \\
              \bottomrule
            \end{tabular}
            \caption{A table caption.}
          \end{table}

          Donec quis posuere ligula. Nunc feugiat elit a mi malesuada consequat. Sed
          imperdiet augue ac nibh aliquet tristique. Aenean eu tortor vulputate,
          eleifend lorem in, dictum urna. Proin auctor ante in augue tincidunt
          tempor. Proin pellentesque vulputate odio, ac gravida nulla posuere
          efficitur. Aenean at velit vel dolor blandit molestie. Mauris laoreet
          commodo quam, non luctus nibh ullamcorper in. Class aptent taciti sociosqu
          ad litora torquent per conubia nostra, per inceptos himenaeos.

          Nulla varius finibus volutpat. Mauris molestie lorem tincidunt, iaculis
          libero at, gravida ante. Phasellus at felis eu neque suscipit suscipit.
          Integer ullamcorper, dui nec pretium ornare, urna dolor consequat libero,
          in feugiat elit lorem euismod lacus. Pellentesque sit amet dolor mollis,
          auctor urna non, tempus sem.

        \end{block}

        \begin{block}{References}

          \nocite{*}
          \footnotesize{\bibliographystyle{plain}\bibliography{poster}}

        \end{block}

      \end{column}

      \separatorcolumn
    \end{columns}
  \end{frame}

\end{document}
