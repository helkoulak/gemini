% Gemini theme
% https://github.com/anishathalye/gemini

\documentclass[final]{beamer}

% ====================
% Packages
% ====================

\usepackage[T1]{fontenc}
\usepackage{lmodern}
\usepackage[size=a0,orientation=portrait,scale=1.45]{beamerposter}
\usetheme{gemini}
\usecolortheme{unamur}
\usepackage{graphicx}
\usepackage{svg}
\usepackage{booktabs}
\usepackage{tikz}
\usepackage{pgfplots}
\pgfplotsset{compat=1.14}
\usepackage{anyfontsize}

% Avoid line breaks in references
\setbeamertemplate{bibliography entry article}{}
\setbeamertemplate{bibliography entry title}{}
\setbeamertemplate{bibliography entry location}{}
\setbeamertemplate{bibliography entry note}{}

% ====================
% Lengths
% ====================

% If you have N columns, choose \sepwidth and \colwidth such that
% (N+1)*\sepwidth + N*\colwidth = \paperwidth
\newlength{\sepwidth}
\newlength{\colwidth}
\setlength{\sepwidth}{0.025\paperwidth}
\setlength{\colwidth}{0.3\paperwidth}

\newcommand{\separatorcolumn}{\begin{column}{\sepwidth}\end{column}}

% ====================
% Title
% ====================

\title{No more Copies! Re-Designing TCPLS Multipath}

\author{Hosam Elkoulak}

\institute[shortinst]{\inst{1} University of Namur}

% ====================
% Footer (optional)
% ====================

\footercontent{
  \href{https://cyberwalingalaxia.be/}{https://cyberwalingalaxia.be/} \hfill
  2024 Cyberwal in Galaxia\hfill
  \href{mailto:hosam.elkoulak@unamur.be}{hosam.elkoulak@unamur.be}}
% (can be left out to remove footer)

% ====================
% Logo (optional)
% ====================

% use this to include logos on the left and/or right side of the header:
% \logoright{\includegraphics[height=7cm]{logo1.pdf}}
% \logoleft{\includegraphics[height=7cm]{logo2.pdf}}

% ====================
% Body
% ====================

\begin{document}

  \begin{frame}[t]
    \begin{columns}[t]
      \separatorcolumn

      \begin{column}{\colwidth}
        \begin{block}{What is TCPLS?}
          TCPLS is an intertwined TLS1.3/TCP design offering transport features like:
          \begin{itemize}
            \item \textbf{Multipath.}
            \item \textbf{Connection migration.}
            \item \textbf{Fail-over.}
          \end{itemize}
          \begin{figure}
            \centering
            \includesvg[width=\textwidth, inkscapelatex=false]{tcpls}
            \caption{The TCPLS network stack.}
            \label{fig:tcpls}
          \end{figure}
        \end{block}

        \begin{block}{Performance Oriented Design}
          TCPLS is designed to enhance overall performance compared to similar protocols like QUIC,
          with a specific emphasis on optimizing the receiving side.
          \begin{itemize}
            \item \textbf{Reverso and zero-copy contiguous receive buffers}
            \item \textbf{Reordering of records in multipath scenario}
            \item \textbf{Size of TCPLS records}
          \end{itemize}

        \end{block}

        \begin{block}{Reverso\cite{rochet2024improvingencryptedtransportprotocol}}

          Two key principles may be applied to any encrypted transport protocol:

          \begin{itemize}
            \item \textbf{Principle 1.} The field order within all encrypted control and data chunks is reversed.
            For a chunk with type (u8), foo (u16), and bar (u64), the order becomes bar (u64), foo (u16), type (u8).
            \item \textbf{Principle 2.} Only a single chunk of data is allowed within a record followed be arbitrary
            number of control chunks.
          \end{itemize}
        \end{block}

        \begin{figure}
          \centering
          \includesvg[width=\textwidth]{unamur}
          \label{fig:unamur}
        \end{figure}
        \begin{figure}
          \centering
          \includesvg[width=\textwidth]{cyberwal}
          \label{fig:cyberwal}
        \end{figure}

      \end{column}

      \separatorcolumn

      \begin{column}{\colwidth}

        \begin{block}{Contiguous Zero-Copy}
          Applying the principles of reverso leads to a simplified receive code path.
          Data received can be decrypted directly into the application buffers avoiding unnecessary copying.
          \begin{figure}
            \centering
            \includesvg[width=\textwidth, inkscapelatex=false]{zero_copy}
            \caption{contiguous decryption in appliction buffers}
            \label{fig:zero-copy}
          \end{figure}
        \end{block}

        \begin{block}{Previous Design Drawbacks}
          The sender's inability to identify the stream to which the received record belongs has resulted in two design drawbacks:
          \heading{Stream Multiplexing}
          Sender has to send the TCPLS' frame \textbf{Stream Change} before switching between streams which consumes
          more bandwidth and requires extra processing for encryption and decryption.
          \begin{figure}
            \centering
            \includesvg[width=\textwidth, inkscapelatex=false]{schange}
            \caption{send stream change frame before swithing streams}
            \label{fig:schange}
          \end{figure}
          \heading{Extra copy along receive path}
          Decryption happens in place and the result is copied in the application receive buffer.
          \begin{figure}
            \centering
            \includesvg[width=\textwidth, inkscapelatex=false]{inplace}
            \caption{Additional copy is required to deliver data in app buffers}
            \label{fig:inplace}
          \end{figure}
        \end{block}


      \end{column}

      \separatorcolumn

      \begin{column}{\colwidth}

        \begin{block}{Solution for Previous Design Drawbacks}
          \heading{TCPLS Header}
          One could solve the two previous issues by adding a header to TCPLS record.
          It consists of 4 bytes encoding the offset where to decrypt in app buffers and 4 bytes encoding the stream ID.
          \begin{figure}
            \centering
            \includesvg[width=0.7\textwidth, inkscapelatex=false]{tcpls_header}
            \caption{TCPLS header}
            \label{fig:header}
          \end{figure}
          As a result two improvements were achieved:
          \begin{itemize}
            \item Remove the need for \textbf{Stream Change} frames,
            as the TCPLS header already contains this information,
            saving bandwidth and eliminating extra encryption/decryption operations for the frame.
            \item TCPLS header is secured separately.
            This makes it possible to decrypt directly in app buffers avoiding the extra copy operation.
            In addition to that the order of the records can be checked before decrypting.
            \begin{figure}
            \centering
            \includesvg[width=0.9\textwidth, inkscapelatex=false]{dec_hdr}
            \caption{Avoid copying by decrypting TCPLS header first and decrypt only records in the right order}
            \label{fig:dec_hdr}
          \end{figure}
          \end{itemize}

       \end{block}

        \begin{block}{References}

          \nocite{*}
          \footnotesize{\bibliographystyle{plain}\bibliography{poster}}

        \end{block}

      \end{column}

      \separatorcolumn
    \end{columns}
  \end{frame}

\end{document}
