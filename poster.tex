% Gemini theme
% https://github.com/anishathalye/gemini

\documentclass[final]{beamer}

% ====================
% Packages
% ====================

\usepackage[T1]{fontenc}
\usepackage{lmodern}
\usepackage[size=a0,orientation=portrait,scale=1.45]{beamerposter}
\usetheme{gemini}
\usecolortheme{unamur}
\usepackage{graphicx}
\usepackage{svg}
\usepackage{booktabs}
\usepackage{tikz}
\usepackage{pgfplots}
\pgfplotsset{compat=1.14}
\usepackage{anyfontsize}

% ====================
% Lengths
% ====================

% If you have N columns, choose \sepwidth and \colwidth such that
% (N+1)*\sepwidth + N*\colwidth = \paperwidth
\newlength{\sepwidth}
\newlength{\colwidth}
\setlength{\sepwidth}{0.025\paperwidth}
\setlength{\colwidth}{0.3\paperwidth}

\newcommand{\separatorcolumn}{\begin{column}{\sepwidth}\end{column}}

% ====================
% Title
% ====================

\title{Some fancy title: followed by some more text}

\author{Hosam Elkoulak}

\institute[shortinst]{\inst{1} University of Namur}

% ====================
% Footer (optional)
% ====================

\footercontent{
  \href{https://cyberwalingalaxia.be/}{https://cyberwalingalaxia.be/} \hfill
   2024 Cyberwal in Galaxia\hfill
  \href{mailto:hosam.elkoulak@unamur.be}{hosam.elkoulak@unamur.be}}
% (can be left out to remove footer)

% ====================
% Logo (optional)
% ====================

% use this to include logos on the left and/or right side of the header:
% \logoright{\includegraphics[height=7cm]{logo1.pdf}}
% \logoleft{\includegraphics[height=7cm]{logo2.pdf}}

% ====================
% Body
% ====================

\begin{document}

  \begin{frame}[t]
    \begin{columns}[t]
      \separatorcolumn

      \begin{column}{\colwidth}

        \begin{block}{TCPLS Handshake}

          A typical TCPLS handshake starts with a TCP handshake followed by a TLS 1.3 handshake.

          \begin{figure}
            \centering
            \includesvg[width=\textwidth]{tcpls_hs}
            \caption{TCPLS handshake}
            \label{fig:tcpls-hs}
          \end{figure}

          TCPLS supports joining additional TCP connections to a TCPLS session.
          To join a new TCP connection, the client sends a fake client hello
          containing one of the tokens received during the TCPLS handshake.

        \end{block}

        \begin{block}{TCPLS Record}

          TCPLS leverages the TLS records to transport data encrypted and authenticated.

          \begin{figure}
            \centering
            \includesvg[width=\textwidth]{tcpls_record}
            \caption{TCPLS Record}
            \label{fig:tcpls-record}
          \end{figure}


        \end{block}
        \begin{block}{TCPLS Transport Services}

          By leveraging TCPLS records and extensions, we can design new
          modern transport services atop the combination of TCP and TLS. Those services are built atop the TCP
          byte stream, and as such they avoid interferences of middleboxes that modify the TCP header.

          \begin{itemize}
            \item \textbf{Stream Multiplexing}
            \item \textbf{Multipath}
            \item \textbf{Failover}
          \end{itemize}
        \end{block}
        \begin{figure}
          \centering
          \includesvg[width=\textwidth]{unamur}
          \label{fig:unamur}
        \end{figure}
        \begin{figure}
          \centering
          \includesvg[width=\textwidth]{cyberwal}
          \label{fig:cyberwal}
        \end{figure}

      \end{column}

      \separatorcolumn

      \begin{column}{\colwidth}

        \begin{block}{Stream Multiplexing}
          Streams can be multiplexed across the available TCP connections of the TCPLS session.
          \begin{figure}
            \centering
            \includesvg[width=\textwidth]{stream_multiplex}
            \caption{Stream Multiplexing}
            \label{fig:stream-multiplex}
          \end{figure}
        \end{block}



        \begin{block}{Failover}
          When a TCP connection fails, e.g., due to middleboxes
          or network outages, TCPLS leverages its joining mechanism to recover the session over another TCP connection.
           \begin{figure}
            \centering
            \includesvg[width=\textwidth, inkscapelatex=false]{failover}
            \caption{Failover}
            \label{fig:failover}
          \end{figure}

        \end{block}

        \begin{block}{Multipath}

          By extending the TLS handshake, TCPLS allows joining several TCP connections to a
          single TCPLS session to support \textbf{connection migration},
          \textbf{stream steering} and \textbf{bandwidth aggregation}.
        \begin{figure}
          \centering
          \includesvg[width=\textwidth]{multipath}
          \caption{Multipath}
          \label{fig:multipath}
        \end{figure}


        \end{block}
        \begin{block}{Performance Oriented Design}
          TCPLS is designed to enhance overall performance compared to similar protocols like QUIC,
          with a specific emphasis on optimizing the receiving side.
          \begin{itemize}
             \item \textbf{Reverso and zero-copy contiguous receive buffers}
            \item \textbf{Reordering of records in multipath scenario}
            \item \textbf{Size of TCPLS records}
          \end{itemize}

        \end{block}

      \end{column}

      \separatorcolumn

      \begin{column}{\colwidth}

        \begin{block}{Reverso and Zero-Copy}

          \heading{Reverso}
          Two key principles may be applied to any encrypted transport protocol:

          \begin{itemize}
            \item \textbf{Principle 1.} The field order within all encrypted control and data chunks is reversed.
            For a chunk with type (u8), foo (u16), and bar (u64), the order becomes bar (u64), foo (u16), type (u8).
            \item \textbf{Principle 2.} Only a single chunk of data is allowed within a record followed be arbitrary
            number of control chunks.
          \end{itemize}
          \heading{Contiguous Zero-Copy}
          Applying the principles of reverso lead to a simplified receive code path.
          Data received can be decrypted directly into the application buffers avoiding unnecessary copying.
          \begin{figure}
            \centering
            \includesvg[width=\textwidth, inkscapelatex=false]{zero_copy}
            \caption{contiguous decryption in appliction buffers}
            \label{fig:zero-copy}
          \end{figure}
        \end{block}


        \begin{block}{References}

          \nocite{*}
          \footnotesize{\bibliographystyle{plain}\bibliography{poster}}

        \end{block}

      \end{column}

      \separatorcolumn
    \end{columns}
  \end{frame}

\end{document}
